\documentclass{article}
\usepackage[utf8]{inputenc}
\usepackage{amsmath}
\usepackage{amsfonts}
\usepackage{setspace}

\title{Physics Lecture }
\author{Philip Policki}
\date{15th October 2020}

\usepackage{natbib}
\usepackage{graphicx}

\begin{document}
\setstretch{1.2}
\maketitle

\section{Passing}
There will be 2 exams / tests.
There will be weekly lists of tasks to do for Exercises. 

\section{Curriculum}
\begin{itemize}
	\item 1/3 of the course is basic logic formulas, useful for simplifying if-else comparisons
	\item 1/3 will cover information what will be helpful to courses on databases, machine learning \& artificial intelligence
	\item 1/3 will cover details on semantics that will be useful on the masters level
\end{itemize}

\section{What is logic?}
\begin{itemize}
	\item Logic is defined as formal apparatus for reasoning.
	\item There are two elements: \\ - Formal language - a set of sentences built with symbols \\ - Semantics - a method of adding meaning to them
\end{itemize}

\section{Sentences in logic}
\begin{itemize}
	\item Basic symbols, variables: a,b,c
	\item Logical connectives, operators:
	\begin{itemize}
		\item OR (alternative, disjunction) $\lor$
		\item AND (conjunction) $\land$
		\item NOT (negation) $\lnot$
		\item IF ... THEN (implication) $\implies$
		\item TRUE IF AND ONLY IF $ \iff $
		\item Tautology $ \top $
	\end{itemize}
\end{itemize}

\section{Logic Laws}
\begin{itemize}
	\item $(a \land b) \lor c \equiv (a \lor c) \land (b \lor c)$
	\item $(a \lor b) \land c \equiv (a \land c) \lor (b \land c)$
	\item $\lnot (a \lor b) \equiv \lnot a \land \lnot b$
	\item $\lnot (a \land b) \equiv \lnot a \lor \lnot b$
\end{itemize}
\end{document}