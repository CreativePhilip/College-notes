\documentclass{article}
\usepackage[utf8]{inputenc}
\usepackage{amsmath}
\usepackage{amsfonts}
\usepackage{setspace}

\title{Physics Lecture }
\author{Philip Policki}
\date{15th October 2020}

\usepackage{natbib}
\usepackage{graphicx}

\begin{document}
\setstretch{1.2}
\maketitle
\tableofcontents
\pagebreak

\section{Introduction}
\subsection{How to pass}
There will be a remote exam, more info nearing the end of the semester.
\subsection{Curriculum}
\setstretch{1}
\begin{enumerate}
	\item Kinematics
	\item Dynamics
	\item Quantum
\end{enumerate}
\setstretch{1.2}

\section{Some math}
$ f'(t) = \frac{df(t)}{dt} = \lim_{\Delta t\to 0} \frac{f(t+\Delta t) - f(t)}{\Delta t}$ \\
Velocity is a derivative of the position vector. \\
Acceleration is a derivative of the velocity vector as well as the second derivative of the position vector. \\

\section{Vector Operations}
\begin{enumerate}
	\item Dot product / Scalar product \\
	$ A \cdot B = |A||B|cos(\alpha)$ \\
	$ A \cdot B = a_x * b_x + a_y * b_y + a_z * b_z$ \\ 
	$ A \cdot A = |A||B|cos(0) = |A|^2$
	\item Vector Scaling \\
	$ kA = (kx_a, ky_a, kz_a)$
\end{enumerate}
\section{Kinematics}
\subsection{Cartesian reference frame}
$f'(x) = \frac{df(x)}{dx} = \lim_{\Delta t\to 0} \frac{f(x + \Delta x)  - f(x)}{\Delta x}$
\begin{itemize}
	\item Position Vector $\vec{r}(x^{(t)}, y^{(t)}, z^{(t)})$
	\item Velocity Vector $\vec{v} = \frac{d\vec{r}}{dt}$ \\
	$|\vec{v}| = \sqrt{\dot{x}^2 + \dot{y}^2 + \dot{z}^2}$
	\item Acceleration Vector $\vec{a} = \frac{d\vec{v}}{dt}$ = $\frac{d}{dt}(\frac{d\vec{r}}{dt}) = \frac{d^2}{dt_2} = \ddot{\vec{r}} = \dot{\vec{v}}$ \\
	$|\vec{a} = \sqrt{\ddot{x}^2 + \ddot{y}^2 + \ddot{z}^2}|$
\end{itemize}
\subsubsection{circular motion}
$\vec{r} = x \hat{n_x} + y \hat{n_y}$ \\
$\vec{v} = \dot{x} \hat{n_x} + \dot{x} \hat{n_Y} $
\subsection{Cylindrical Reference Frame}
Helps to simplify motions like circular, spiral in 2d or 3d etc.
Coordinates in the rf:
\begin{itemize}
	\item z = height
	\item $\rho$ magnitude of the radius
	\item $\phi$ projection angle of the radius
\end{itemize}

\subsection{Normal Reference Frame}
\section{Dynamics}
Concerns the cause of motion
Physical laws of dynamics(Newtonian Dynamics) \\
\subsection{3 Newton Laws}
\begin{enumerate}
	\item If there is no force acting on an object, the object stays at rest or moves with a constant velocity
	\item In a inertial R-F:  $\vec{F} = m\vec{a}$
	\item When one body exerts a force on a second body, the second body simultaneously exerts a force equal in magnitude and opposite in direction on the first body.
\end{enumerate}
%\pagebreak
\subsection{4 Fundamental Forces of Nature}
\begin{enumerate}
	\item Gravity $\vec{F_g} = -G\frac{Mm}{r^2}$; $G\approx 6.67 * 10^{-11}[N\frac{m^2}{kg^2}]$
	\item Electromagnetic $\vec{F_c} = \frac{1}{4\pi\epsilon_0}\frac{Qq}{r^2}$
	\item Weak
	\item Strong
\end{enumerate}
\end{document}