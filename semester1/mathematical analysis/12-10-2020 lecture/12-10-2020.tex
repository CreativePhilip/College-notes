\documentclass{article}
\usepackage[utf8]{inputenc}
\usepackage{amsmath}
\usepackage{amsfonts}
\usepackage{setspace}

\title{Mathematical Analysis \\ Lecture }
\author{Philip Policki}
\date{12th October 2020}

\usepackage{natbib}
\usepackage{graphicx}

\begin{document}
\setstretch{1.2}
\maketitle
\section{Rules}
	No textbook, so take notes. \\
	Classes are mandatory
	
\section{Requirements}
	During the classes we will start with a quiz, every practice.
	To pass the course you need $50\%$ of points from the quizzed. A Quizes is 15min every quiz is worth 5 points. You get points from your top 10 quizes.
\section{Notation}
\subsection{Number sets}
\begin{enumerate}
	\item Natural Numbers $ N = \{1, 2, 3, ...\}$ 
	\item Integers $ \mathbb{Z} = \{..., -2, -1, 0, 1, 2, ...\}$
	\item Rational $ \mathbb{Q} = \{ \frac{p}{q}; p, q \in Z, q\neq0 \} $
	\item Irrational $ ex.: \sqrt{2}, \pi, ... $ 
	\item Real Numbers $\mathbb{R} = Rational + Rational $
\end{enumerate}

\subsection{Sets notation}
$(a, b) = x \in R: a<x<b$ \\
$(a, b) = x \in R: a \leq x \leq b$ \\	
$(a, \infty) = x\in R: x > a$ \\ \\
$(a, b) - open interval$ \\
$ [a, b] - closed interval $ \\ \\
$ A \subset B$ A is a subset of B \\
$ x\in A$ X is an element of A, x belongs to A \\
$ x\notin A$ X is not an element of A, x does not belong to A \\

\subsection{Cartesian Product}
Given two sets A and B, we can form the set consisting of all ordered pairs of the form $(a, b)$ where $a \in A$ and $b \in B$. This set is called the Cartesian product of A and B and is denoted by $AxB$\\
$AxB \{(a, b): a\in A, \in B \}$ \\
If $A=B$, then $AxA$ is denoted by $A^2$\\

\subsection{Quantifiers}
\begin{enumerate}
	\item  Existential $\exists$ "There exists x such that", "For at least one x"
	\item Universal $\forall$ "For all x", "For each x", "For every x"
\end{enumerate}

Example: \\
$ \exists \; t > 0 \; \forall \; x\in \mathbb{R} \; x^2 + 4x + 4 > t $ \\
The statement above is false \\
The negation of the statement: \\
$ \forall \; t>0 \; \exists \; x_0 \in \mathbb{R} \; x_0^2 + 4x_0 + 4 \leq t $

\section{Functions}
A function f is a rule that assigns to each element x in a set A \underline{exactly one} element , called f(x), in a set B. \\
In our class $ A \subset \mathbb{R} $ and $ B \subset \mathbb{R}$
The set A is called the domain of the function f and will be denoted $D_f$.\\
The range of the function f is the set of all possible values of f(x) as x varies throughout the domain. The range of f will be denoted by $R_f$.\\
The most common method for visualizing a function is its.\\ 
If f is a function with domain $D_f$ then its graph is the set of ordered pairs. \\
$ \{(x, y) \in \mathbb{R}^2 \; x\in D_f, y=f(x) \}$ \\ \\
Example: \\ 
Min function  \\
$f(x) = min\{x, x^2\}$ \\
$f(2) = min\{2, 4\}$ = 2 \\
$f(\frac{1}{2}) = min\{\frac{1}{2}, \frac{1}{4}\}$ = $\frac{1}{4}$ \\ \\ \\
Absolute \\
$f(x) = |x| = \{x, if x \geq 0 \; or -x, if x-2 < 0\}$\\
$f(x) = |x -2| = \{x-2, if x \geq 0 \; or -(x-2), if x-2 < 0\}$ \\ \\
$|x-a|$ represents the distance between x and a
\end{document}