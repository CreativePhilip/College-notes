\documentclass{article}
\usepackage[utf8]{inputenc}
\usepackage{amsmath}
\usepackage{amsfonts}
\usepackage{setspace}

\title{Mathematical Analysis \\ Lecture }
\author{Philip Policki}
\date{12th October 2020}

\usepackage{natbib}
\usepackage{graphicx}

\begin{document}
\setstretch{1.2}
\maketitle
\tableofcontents
\pagebreak
\section{Rules}
	No textbook, so take notes. \\
	Classes are mandatory
	
\section{Requirements}
	During the classes we will start with a quiz, every practice.
	To pass the course you need $50\%$ of points from the quizzed. A Quizes is 15min every quiz is worth 5 points. You get points from your top 10 quizes.
\section{Notation}
\subsection{Number sets}
\begin{enumerate}
	\item Natural Numbers $ N = \{1, 2, 3, ...\}$ 
	\item Integers $ \mathbb{Z} = \{..., -2, -1, 0, 1, 2, ...\}$
	\item Rational $ \mathbb{Q} = \{ \frac{p}{q}; p, q \in Z, q\neq0 \} $
	\item Irrational $ ex.: \sqrt{2}, \pi, ... $ 
	\item Real Numbers $\mathbb{R} = Rational + Irrational $
\end{enumerate}

\subsection{Sets notation}
$(a, b) = x \in R: a<x<b$ \\
$(a, b) = x \in R: a \leq x \leq b$ \\	
$(a, \infty) = x\in R: x > a$ \\ \\
$(a, b) - open interval$ \\
$ [a, b] - closed interval $ \\ \\
$ A \subset B$ A is a subset of B \\
$ x\in A$ X is an element of A, x belongs to A \\
$ x\notin A$ X is not an element of A, x does not belong to A \\

\subsection{Cartesian Product}
Given two sets A and B, we can form the set consisting of all ordered pairs of the form $(a, b)$ where $a \in A$ and $b \in B$. This set is called the Cartesian product of A and B and is denoted by $AxB$\\
$AxB \{(a, b): a\in A, \in B \}$ \\
If $A=B$, then $AxA$ is denoted by $A^2$\\

\subsection{Quantifiers}
\begin{enumerate}
	\item  Existential $\exists$ "There exists x such that", "For at least one x"
	\item Universal $\forall$ "For all x", "For each x", "For every x"
\end{enumerate}

Example: \\
$ \exists \; t > 0 \; \forall \; x\in \mathbb{R} \; x^2 + 4x + 4 > t $ \\
The statement above is false \\
The negation of the statement: \\
$ \forall \; t>0 \; \exists \; x_0 \in \mathbb{R} \; x_0^2 + 4x_0 + 4 \leq t $

\section{Functions}
A function f is a rule that assigns to each element x in a set A \underline{exactly one} element , called f(x), in a set B. \\
In our class $ A \subset \mathbb{R} $ and $ B \subset \mathbb{R}$
The set A is called the domain of the function f and will be denoted $D_f$.\\
The range of the function f is the set of all possible values of f(x) as x varies throughout the domain. The range of f will be denoted by $R_f$.\\
The most common method for visualizing a function is its.\\ 
If f is a function with domain $D_f$ then its graph is the set of ordered pairs. \\
$ \{(x, y) \in \mathbb{R}^2 \; x\in D_f, y=f(x) \}$ \\ \\
Example: \\ 
Min function  \\
\begin{equation}
	\begin{gathered}
		f(x) = min\{x, x^2\} \\
		f(2) = min\{2, 4\} = 2 \\
		f(\frac{1}{2}) = min\{\frac{1}{2}, \frac{1}{4}\} = \frac{1}{4} \\ \\ \\
	\end{gathered}
\end{equation}

Absolute \\
\begin{equation}
	\begin{gathered}
		f(x) = |x| = \{x, if x \geq 0 \; or -x, if x-2 < 0\}\\
		f(x) = |x -2| = \{x-2, if x \geq 0 \; or -(x-2), if x-2 < 0\} \\ \\
	\end{gathered}
\end{equation}

$|x-a|$ represents the distance between x and a
\subsection{The Vertical Line Test}
A curve is the XY plane is the graph of a function of x if and only if no vertical line intersects the curve more than once.
\subsection{Classes of functions}
\begin{enumerate}
	\item Periodic functions \\
	We say that f is a periodic function if \\ 
	$\exists T>0 \; \forall x\in D_f(x \pm T \in D_f$ and $f(x+T) = f(x))$ \\
	A periodic function is a function that repeats its values after some determined period has been added to its independent variable.\\
	\item Symmetric functions
	\begin{itemize}
		\item Even\\
		A function f is called even if: \\
		$\forall x \in D_f \; (-x  \in D_f)$ and $f(-x) = f(x)$ \\
		The geometric  significance of an even function is that its graph is symmetric with respect to the Y axis. \\
		If f is even $D_f$ is symmetric about the Y Axis.
		\item Odd \\ 
		A function f is called odd if: \\
		$ \forall x \in D_f (-x \in D_f) $ and $f(-x) = -f(x)$\\
		The graph of and odd function is symmetric about the origin.\\
		If an odd function is defined at x=0 then f(0) must be 0!!\\ \\
		Example:
		Check if function is even or odd.\\
		$f(x) = \frac{3^x - 3^{-x}}{x}$
		\begin{enumerate}
			\item Check if domain is symmetric \\
			$D_f = \mathbb{R}\setminus \{0\}$
			\item Substitute -x for x
		\end{enumerate}
	\item Monotonicity: \\
	A function is monotonic if it is increasing, or decreasing, or non-decreasing, or non-increasing.
	\begin{itemize}
		\item Increasing:\\
		A function f is called increasing on a set $I \subset D_f$, \\
		if $\forall x_1, x_2 \in I \; [(x_1 < x_2) \Rightarrow (f(x_1) < f(x_2))]$
		\item Non-decreasing: \\
		A function f is called increasing on a set $I \subset D_f$, \\
		if $\forall x_1, x_2 \in I \; [(x_1 < x_2) \Rightarrow (f(x_1) \leq f(x_2))]$
		\item Decreasing: \\
		A function f is called increasing on a set $I \subset D_f$, \\
		if $\forall x_1, x_2 \in I \; [(x_1 < x_2) \Rightarrow (f(x_1) > f(x_2))]$
		\item Non-increasing: \\
		A function f is called increasing on a set $I \subset D_f$, \\
		if $\forall x_1, x_2 \in I \; [(x_1 < x_2) \Rightarrow (f(x_1) \geq f(x_2))]$
	\end{itemize}
	Algebraic way to check monotonicity: \\
	let $f(x) = \frac{1}{1+x^2}$ and $I = (-\infty, 0]$ \\
	Take any 2 points $x_1, x_2 \in I$ with $x_1 < x_2$. \\
	$f(x_2) - f(x_1) = \frac{1}{1+x_2^2} - \frac{1}{1+x_1^2}$ = \\
	$\frac{1+x_1^2 - 1 + x_2^2}{(1+x_1^2)(1+x_2^2)}$ = \\
	$\frac{(x_1-x_2)(x_1+x_2)}{(1+x_2^2)(1+x_1^2)}$ \\
	$x_1 - x_2  < 0$ \\
	$x_1 + x_2 < 0$ \\
	\end{itemize}
	\subsection{New functions from Old functions}
	\begin{itemize}
		\item Vertical and horizontal shifts: \\
		Suppose $c>0$.\\ To obtain the graph of $y = f(x) + c$ shift the graph of y a distance of c units upwards. If $y = f(x)-c$ shift downwards. \\ \\
		To obtain the graph of $y=f(x-c)$ shift the graph y a distance of c units to the right. \\
		To obtain the graph of $y=f(x+c)$ shift the graph y a distance of c units to the left. \\
		\item Vertical and horizontal stretching and reflecting: \\
		Suppose $c>1$. \\
		To obtain the graph $y=c*f(x)$ stretch y vertically by a factor of x. \\
		To obtain the graph $y=f(c*x)$ compress the graph of y horizontally by a factor of c. \\
		To obtain the graph $y=-f(x)$ reflect the graph of y about the x axis. \\
		To obtain the graph $y=f(-x)$ reflect the graph of y about the y axis. \\
		\item Algebra of functions: \\
		let f and g be functions with domains $D_f$ and $D_g$. Then the functions $f_g$, $f-g$, $fg$ and $\frac{f}{g}$ are as follows: \\
		$(f\pm g)(x) = f(x)\pm g$; $D_{f+g} = D_f \cap D_g$ \\ 
		$(f* g)(x) = f(x)*g(x)$; $D_{f*g} = D_f \cap D_g$\\
		$(\frac{f}{g})(x) = \frac{f(x)}{f(g)}$; $D_{f+g} = D_f \cap D_g$
	\end{itemize}
	
\end{enumerate}
\end{document}